\documentclass[11pt,a4paper,notitlepage]{article}

\usepackage[T1]{fontenc}
\usepackage[utf8]{inputenc}
\usepackage[english]{babel}
\usepackage{fullpage}
\usepackage{amsmath}
\usepackage{amsfonts}
\usepackage{amssymb}
\usepackage{verbatim}
\usepackage{listings}
\usepackage{color}
\usepackage{setspace}
\usepackage{epstopdf}
\usepackage{graphicx}
\usepackage{caption}
\usepackage{subcaption}
\usepackage{float}
\usepackage{epstopdf}
\usepackage{hyperref}
\usepackage{braket}
\pagenumbering{arabic}

\definecolor{codepurple}{rgb}{0.58,0,0.82}
\definecolor{backcolour}{rgb}{0.95,0.95,0.92}
\definecolor{dkgreen}{rgb}{0,0.6,0}
\definecolor{gray}{rgb}{0.5,0.5,0.5}
\definecolor{mauve}{rgb}{0.58,0,0.82}
%\setlength{\parindent}{0pt}

\lstdefinestyle{pystyle}{
  language=Python,
  aboveskip=3mm,
  belowskip=3mm,
  columns=flexible,
  basicstyle={\small\ttfamily},
  backgroundcolor=\color{backcolour},
  commentstyle=\color{dkgreen},
  keywordstyle=\color{magenta},
  numberstyle=\tiny\color{gray},
  stringstyle=\color{codepurple},
  basicstyle=\footnotesize,  
  breakatwhitespace=false
  breaklines=true,
  captionpos=b,
  keepspaces=true,
  numbers=left,
  numbersep=5pt,
  showspaces=false,
  showstringspaces=false,
  showtabs=false,
  tabsize=2
}
\lstdefinestyle{iStyle}{
  language=IDL,
  aboveskip=3mm,
  belowskip=3mm,
  columns=flexible,
  basicstyle={\small\ttfamily},
  backgroundcolor=\color{backcolour},
  commentstyle=\color{dkgreen},
  keywordstyle=\color{magenta},
  numberstyle=\tiny\color{gray},
  stringstyle=\color{codepurple},
  basicstyle=\footnotesize,  
  breakatwhitespace=false
  breaklines=true,
  captionpos=b,
  keepspaces=true,
  numbers=left,
  numbersep=5pt,
  showspaces=false,
  showstringspaces=false,
  showtabs=false,
  tabsize=2
}
\lstdefinestyle{c++style}{
  language=C++,
  keywordstyle=\color{blue}\ttfamily,
  stringstyle=\color{red}\ttfamily,
  commentstyle=\color{green}\ttfamily,
  morecomment=[l][\color{magenta}]{\#}
  aboveskip=3mm,
  belowskip=3mm,
  columns=flexible,
  basicstyle={\small\ttfamily},
  backgroundcolor=\color{backcolour},
  numberstyle=\tiny\color{gray},
  basicstyle=\footnotesize,  
  breakatwhitespace=false
  breaklines=true,
  captionpos=b,
  keepspaces=true,
  numbers=left,
  numbersep=5pt,
  showspaces=false,
  showstringspaces=false,
  showtabs=false,
  tabsize=2
}

\title{\normalsize Fys4150: Computational Physics \\
\vspace{10mm}
\huge Project 2\\
\vspace{10mm}
\normalsize Due date {\bf October $3^{rd}$, 2016}}

% Skriv namnet ditt her og fjern kommenteringa
\author{Øyvind B. Svendsen, Magnus Christopher Bareid \\ un: oyvinbsv, magnucb}

\newcommand{\SE}{Schr\"odinger equation}
\newcommand{\laplacian}{\vec{\nabla}^2}
%\newcommand{\hbar}{\bar{h}}
\newcommand\pd[2]{\frac{\partial #1}{\partial #2}}
\def\doubleunderline#1{\underline{\underline{#1}}}


\begin{document}
\noindent
\maketitle
\vspace{10mm}
\begin{abstract}
\end{abstract}

\begin{center}
%\line(1,0){450}
\end{center}

\newpage
\tableofcontents

\newpage
\section{Introduction}
The goal of this project is to solve the \SE for one and two electrons in, and not in, an harmonic oscillator potential. The \SE will be simplified to an eigenvalue problem in one dimension(radial). 

The structure of the report will first lay out all the theory behind solving this problem, and all the subproblems that arissen. Afterwards the theory will be turned into a viable, realistic, implementation in the method's section, before presenting the results. 

Before any finishing remarks, the numerical programs and algorithms will be discussed. Considering both numerical advantages and disadvantages, before presenting the programs themselves. For any individual interested in replicating the solution; the full projectfolder can be found on Github.

\section{Theory}
%Schrodinger equation
Beginning with the time-independent \SE for one electron:
\begin{equation} \label{eq:SE 1e}
	-\frac{\hbar^2}{2m}\laplacian \psi + V\psi = E\psi 			
	\indent
	\cite[chapter 4, p.132, eq:4.8]{schrodinger_equation}
\end{equation}

This equation will be hammered to a more solvable one. Firstly eq.\eqref{eq:SE 1e} will be transformed to radial coordinates with spherical harmonics.
\begin{align*}
	-\frac{\hbar^2}{2m}\left( \frac{1}{r^2} \frac{d}{dr}\left [ r^2\frac{d}{dr}\right ] - \frac{l(l+1)}{r^2}\right) \psi(r) + V(r)\psi(r) = E\psi(r)
\end{align*}

\begin{flushright}
\begin{minipage}{0.5\linewidth}
	Perform change of variables 
	\begin{align*}
		\psi(r) &= \frac{u(r)}{r} \\
		\frac{d\psi}{dr} &= \frac{d}{dr}\left[\frac{u(r)}{r}\right] \\
		&= \frac{1}{r}\frac{du}{dr} - \frac{u}{r^2} \\
		\frac{d}{dr}\left[ r^2 \frac{d\psi}{dr} \right] 
		&= \frac{d}{dr}\left[ r\frac{du}{dr} - u(r) \right] \\
		&= r\frac{d^2u}{dr^2}
	\end{align*}
\end{minipage}
\end{flushright}

\begin{align*}
	-\frac{\hbar^2}{2m}\left( \frac{1}{r} \frac{d^2u}{dr^2} - \frac{l(l+1)u(r)}{r^3}\right) + V(r)\frac{u(r)}{r} &= E\frac{u(r)}{r} \\
	-\frac{\hbar^2}{2m}\left( \frac{d^2u}{dr^2} - \frac{l(l+1)u(r)}{r^2}\right) + V(r)u(r) &= Eu(r)
\end{align*}

\begin{flushright}
\begin{minipage}{0.5\linewidth}
	Perform change of variables 
	\begin{align*}
		r &= \alpha \rho \\
		\frac{d\rho}{dr} &= \frac{1}{\alpha} \\
		\frac{du}{dr} &= \frac{d\rho}{dr}\frac{du}{d\rho} = \frac{1}{\alpha}\frac{du}{d\rho} \\
		\frac{d^2u}{dr^2} &= \left(\frac{d\rho}{dr}\right)^2 \frac{d^2u}{d\rho^2} = \frac{1}{\alpha^2} \frac{d^2u}{d\rho^2}
	\end{align*}
\end{minipage}
\end{flushright}

\begin{align*}
	-\frac{\hbar^2}{2m\alpha^2}\frac{d^2u}{d\rho^2} + \left(  \frac{\hbar^2}{2m\alpha^2}\frac{l(l+1)}{\rho^2} + V(\rho)\right)u(\rho) &= Eu(\rho) 
	\intertext{Now we will look for specific cases of $l=0$ and 
	$V(\rho)=\frac{1}{2}k(\alpha\rho)^2$(harmonic oscilator potential)}
	-\frac{\hbar^2}{2m\alpha^2}\frac{d^2u}{d\rho^2} + \frac{1}{2}k(\alpha\rho)^2 u(\rho) &= Eu(\rho) \\ 
	-\frac{d^2u}{d\rho^2} + \frac{mk\alpha^4\rho^2}{\hbar^2} u(\rho) &= \frac{2m\alpha^2}{\hbar^2}Eu(\rho)
	\intertext{by saying that $\frac{mk\alpha^4\rho^2}{\hbar^2}=1 \Rightarrow\alpha=\sqrt[4]{\frac{\hbar^2}{mk}}$ and $\frac{2m\alpha^2}{\hbar^2}E=\lambda$ the equation becomes quite pretty (remember that $\alpha$ is merely a scale factor that are determined by us).}
	-\frac{d^2u}{d\rho^2} + \rho^2u(\rho) &= \lambda u(\rho)
\end{align*}

\begin{minipage}{0.5\linewidth}
	This can be written as a matrix-equation:
	\begin{equation} \label{eq:matrix}
		\hat{A} \vec{u} = \lambda \vec{u}
	\end{equation}
\end{minipage} 
\begin{minipage}{0.5\linewidth}
	%Where:
	\begin{align*}
		\hat{A} &= -\frac{d^2}{d\rho^2} + \rho^2 \\
		\vec{u} &= u(\rho) 
	\end{align*}
\end{minipage}


\subsection{Discretization of matrix-equation}
	In the Linear Algebra chapter\cite[Jensen 2016] {discretize_double_deriv} the discretized function for the double derivative is given by:
	\begin{equation}\label{eq:discretized}
		y_i^{''} = \frac{y_{i+1} - 2y_i + y_{i-1}}{h^2} + O(h^2)
	\end{equation}
	Where; $x_i = x_0 + ih$, $y_i = y(x_i)$, where h is the steplength of x, $y^{''}$ is the double derivative of y and $O(h^2)$ is the mathematical error of this approximation. \\
	As shown in the Linear Algebra chapter this can be turned into a matrix-operation of the differential equation.
	\begin{align*}
		-\frac{d^2u_i}{d\rho^2} + \rho_i^2u_i &= \lambda_i u_i
		\intertext{ \flushright \small 
			Where $u_i = u(\rho_i)$ for discretized lengths, with 
			corresponding eigenvalues, $\lambda_i$
		}
		\eqref{eq:discretized} \Rightarrow 
		-\frac{u_{i+1} - 2u_i + u_{i-1}}{h^2} + \rho_i^2u_i 
		&\simeq \lambda_i u_i
		\intertext{\flushright \small 
		Using Einstein notation for discretized vectors
		}
		\hat{A} \vec{u} &\simeq \lambda \vec{u} \\
		\eqref{eq:matrix} \Rightarrow 
		\hat{A} &= 
			\begin{bmatrix}
				\frac{2}{h^2} + \rho_0^2 & \frac{-1}{h^2} & 0 & \hdots & & 0\\
				\frac{-1}{h^2} & \frac{2}{h^2} + \rho_1^2 & \frac{-1}{h^2} & & & \vdots \\
				0 & \frac{-1}{h^2} & \frac{2}{h^2} + \rho_2^2 &  & & &\\
				\vdots & & & \ddots & &\\
				& & & & \ddots & \frac{-1}{h^2} \\
				0 & \hdots & & & \frac{-1}{h^2} & \frac{2}{h^2} + \rho_N^2 \\				
			\end{bmatrix} \\ 
		\eqref{eq:matrix} \Rightarrow
		\vec{u} &= u_i = 
			\begin{bmatrix}
				u_0 \\ u_1 \\ \vdots \\ u_N
			\end{bmatrix}
	\end{align*}

\subsection{Jacobi's Method}
In order to find the eigenvalues of a matrix, the matrix matrix must be reduced to a diagonal matrix. If this is done and orthogonality is preserved at the same time, the diagonal elements of the very same matrix will become it's own eigenvalues. 

By establishing a unitary matrix, S, that maintains orthogonality a new matrix can be found.
I.o.w S is a rotational matrix in the plain.
\begin{align*}
	S = \begin{bmatrix}
		 1 & & & & &  \\
		 & \ddots & & & & \\
		 & & \cos(\theta)& \hdots & \sin(\theta)& & \\
		 & & \vdots & \ddots & \vdots & & \\
		 & & -\sin(\theta)& \hdots & \cos(\theta) & & \\
		 & & & & & \ddots & \\
		 & & & & & & 1 \\
	\end{bmatrix}
\end{align*}
%TODO
\subsection{Orthogonality}
%TODO
\section{Method} \label{sec:method}
%TODO	
\section{Results}
\section{Explanation of programs}
\subsection{Unit tests}
%TODO
%orthogonality
%symmetry
\subsection{main.cpp}
\section{Conclusion and discussion}
\section{Appendix - Github} \label{section:github}
\url{https://github.com/theknight1509/FYS4150_Project2}
\section{References}
We got a lot of inspiration from the example-code in \cite[web-site]{example_code}

\begin{thebibliography}{9}

\bibitem{example_code}
  Jensen, M 2016,
  program1.cpp,
  viewed $29^{th}$ September 2016,
  \url{https://github.com/CompPhysics/ComputationalPhysics/blob/master/doc/Programs/LecturePrograms/programs/EigenProblems/cpp/program1.cpp}.
\bibitem{schrodinger_equation}
  Griffiths D.J, 
  Pearson (1995),
  \emph{Introduction to quantum mechanics},
  $2^{nd}$ edition(international).
  
\bibitem{discretize_double_deriv}
	Jensen, M 2016, 
	Computational Physics Lectures: Linear
Algebra methods, 
	viewed $27^{th}$ September 2016, 
	\url{http://compphysics.github.io/ComputationalPhysics/doc/pub/linalg/pdf/linalg-print.pdf} p.12. 

\end{thebibliography}
\end{document}
